
\section{\module{plasTeX.TeX} --- The \TeX\ Stream}

\declaremodule{standard}{plasTeX.TeX}
\modulesynopsis{The \TeX\ stream processor}

The \TeX\ stream is the piece of \plasTeX\ where the parsing of the
\LaTeX\ document takes place.  While the \class{TeX} class is fairly
large, there are only a few methods and attributes designated in the
public API.  

The \TeX\ stream is based on a Python generator.  When you feed it
a \LaTeX\ source file, it processes the file much like \TeX\ itself.
However, on the output end, rather than a DVI file, you get a 
\plasTeX\ document object.  The basic usage is shown in the code below.
\begin{verbatim}
from plasTeX.TeX import TeX
doc = TeX(file='myfile.tex').parse()
\end{verbatim}

\subsection{TeX Objects}

\begin{classdesc}{TeX}{\optional{ownerDocument, file}}
The \class{TeX} class is the central \TeX\ engine that does all of the 
parsing, invoking of macros, and other document building tasks.
You can pass in an owner document if you have a customized document
node, or if it contains a customized configuration; otherwise, 
the default \class{TeXDocument} class is instantiated.  The \var{file}
argument is the name of a \LaTeX\ file.  This file will be 
searched for using the standard \LaTeX\ technique and will be 
read using the default input encoding in the document's configuration.
\end{classdesc}

\begin{methoddesc}[TeX]{disableLogging}{}
disables logging.  This is useful if you are using the \class{TeX} object
within another library and do not want all of the status information to 
be printed to the screen.

\note{This is a class method.}
\end{methoddesc}

\begin{memberdesc}[TeX]{filename}
the current filename being processed
\end{memberdesc}

\begin{memberdesc}[TeX]{jobname}
the name of the basename at the top of the input stack
\end{memberdesc}

\begin{memberdesc}[TeX]{linenumber}
the line number of the current file being processed 
\end{memberdesc}


\begin{methoddesc}[TeX]{expandtokens}{tokens, normalize=False}
expand a list of unexpanded tokens.  This method can be used to expand
tokens without having them sent to the output stream.  The returned value
is a \class{TeXFragment} populated with the expanded tokens.
\end{methoddesc}

\begin{methoddesc}[TeX]{input}{source}
add a new input source to the input stack.  \var{source} should be
a Python file object.  This can be used to add additional input sources
to the stream after the \class{TeX} object has been instantiated.
\end{methoddesc}

\begin{methoddesc}[TeX]{__iter__}{}
return a generator that iterates through the tokens in the source.  
This method allows you to treat the \class{TeX} stream as an iterable
and use it in looping constructs.  While the looping is generally
handled in the \method{parse()} method, you can manually expand the
tokens in the source by looping over the \class{TeX} object as well.
\begin{verbatim}
for tok in TeX(open('myfile.tex')):
    print tok
\end{verbatim}
\end{methoddesc}

\begin{methoddesc}[TeX]{itertokens}{}
return an iterator that iterates over the unexpanded tokens in the 
input document.
\end{methoddesc}

\begin{methoddesc}[TeX]{kpsewhich}{name}
locate the given file in a kpsewhich-like manner.  The full path to the
file is returned if it is found; otherwise, \var{None} is returned.
\note{Currently, only the directories listed in the environment variable
\envvar{TEXINPUTS} are searched.}
\end{methoddesc}

\begin{methoddesc}[TeX]{normalize}{tokens}
joins consecutive text tokens into a string.  If the list of tokens contain
tokens that are not text tokens, the original list of tokens is returned.
\end{methoddesc}

\begin{methoddesc}[TeX]{parse}{output=None}
parse the sources currently in the input stack until they are empty.
The \var{output} argument is an optional \class{Document} node to put
the resulting nodes into.  If none is supplied, a \class{TeXDocument}
instance will be created.  The return value is the document from the
\var{output} argument or the instantiated \class{TeXDocument} object.
\end{methoddesc}

\begin{methoddesc}[TeX]{pushtoken}{token}
pushes a token back into the input stream to be re-read.
\end{methoddesc}

\begin{methoddesc}[TeX]{pushtokens}{tokens}
pushes a list of tokens back into the input stream to be re-read.
\end{methoddesc}

\begin{methoddesc}[TeX]{readArgument}{*args, **kwargs}
parse a macro argument without the \LaTeX\ source that created it.
This method is just a thin wrapper around \method{readArgumentAndSource}.
See that method for more information.
\end{methoddesc}

\begin{methoddesc}[TeX]{readArgumentAndSource}{spec=None, subtype=None, 
       delim=',', expanded=False, default=None, parentNode=None, name=None}
parse a macro argument.  Return the argument and the \LaTeX\ source that
created it.  The arguments are described below.

\begin{tableii}{l|p{4in}}{var}{Option}{Description}
\lineii{spec}{string containing information about the type of
            argument to get.  If it is 'None', the next token is
            returned.  If it is a two-character string, a grouping
            delimited by those two characters is returned (i.e. '[]').
            If it is a single-character string, the stream is checked
            to see if the next character is the one specified.  
            In all cases, if the specified argument is not found,
            'None' is returned.}
\lineii{type}{data type to cast the argument to.  New types can be
            added to the self.argtypes dictionary.  The key should
            match this 'type' argument and the value should be a callable
            object that takes a list of tokens as the first argument
            and a list of unspecified keyword arguments (i.e. **kwargs)
            for type specific information such as list delimiters.}
\lineii{subtype}{data type to use for elements of a list or dictionary}
\lineii{delim}{item delimiter for list and dictionary types}
\lineii{expanded}{boolean indicating whether the argument content
            should be expanded or just returned as an unexpanded 
            text string}
\lineii{default}{value to return if the argument doesn't exist}
\lineii{parentNode}{the node that the argument belongs to}
\lineii{name}{the name of the argument being parsed}
\end{tableii}

The return value is always a two-element tuple.  The second value is always
a string.  However, the first value can take the following values.

\begin{tableii}{l|l}{}{Value}{Condition}
\lineii{None}{the requested argument wasn't found}
\lineii{object of requested type}{if \var{type} was specified}
\lineii{list of tokens}{all other arguments}
\end{tableii}

\end{methoddesc}

\begin{methoddesc}[TeX]{source}{tokens}
return the \LaTeX\ representation of the tokens in \var{tokens}
\end{methoddesc} 

\begin{methoddesc}[TeX]{texttokens}{text}
convert a string of text into a series of tokens
\end{methoddesc}



